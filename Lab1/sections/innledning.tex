\section{Innledning}

Utviklingen av elektrisk kommunikasjon på 1800-tallet skapte et behov for å forstå hvordan 
signaler forplantet seg i lange ledningsnettverk. Telegrafen, som var den første teknologien 
for rask kommunikasjon over store avstander, ble raskt bygget ut både i Europa og USA. Etter hvert 
som ledningsnettene ble lengre, oppsto praktiske problemer: signalene ble svakere, de ble 
forsinket, og de kunne endre form. For å løse disse utfordringene måtte man utvikle en matematisk 
modell som tok hensyn til de fysiske egenskapene til en elektrisk ledning.
\\[1em]
Oliver Heaviside, en britisk fysiker og ingeniør, brukte Maxwells ligninger som utgangspunkt og 
formulerte på slutten av 1800-tallet det som i dag kalles \textit{telegrafligningen}. Den beskriver 
hvordan spenning og strøm endrer seg i tid og rom langs en elektrisk linje, idét 
resistans, induktans, kapasitans og konduktans inkluderes i modellen. Telegrafligningen ble et viktig teoretisk verktøy 
for å forstå og forbedre kvaliteten på telegraf- og telefonlinjer, og den la grunnlaget for hele 
transmisjonslinjeteorien som fortsatt brukes i moderne elektronikk og telekommunikasjon \cite{geeksforgeeks_telegrapher}.
\\[1em]
I dag er de samme prinsippene fortsatt aktuelle. Ethernetkabler, koaxialkabler og telefonlinjer kan modelleres 
som transmisjonslinjer, der signalet påvirkes av tap og dispersjon. Ethernet-standarden setter en 
maksimal kabellengde på 100 meter for å sikre at signalet kan overføres uten for store tap og 
forvrengninger. Koaxialkablers maksimale lengde uten stort tap av signalkvalitet vil også variere basert på frekvensene 
som brukes. På samme måte mister lange telefonkabler kvalitet, spesielt når de brukes til høyere 
frekvenser som i bredbånd (DSL). Begge disse fenomenene kan forklares gjennom telegrafligningen.
\\[1em]
I denne rapporten tar vi derfor utgangspunkt i telegrafligningen som matematisk modell og gjennom numerisk
modellering og laboratorietester, undersøker vi hvordan signaler oppfører seg i korte og 
lange ledningsnettverk ved forskjellige frekvenser. Vi vil også se på hvordan ulike parametere påvirker 
signalets kvalitet, med demping, faseendring og forvrengning som sentrale temaer. Dette gir oss en bedre forståelse av
hvordan elektriske signaler oppfører seg i praksis. 