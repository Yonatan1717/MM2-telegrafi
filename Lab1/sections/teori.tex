
\section{Teori}

    \textbf{Motivasjon og sammenheng:} For å forstå hvorfor det er en grense på 100 meter for Ethernet-kabler, og hvordan signaler dempes og forvrenges i slike kabler, bruker vi telegrafligningen. Denne beskriver hvordan elektriske signaler forplanter seg i transmisjonslinjer, og gir oss verktøyene til å analysere tap, dispersjon.

\subsection{Kirchhoffs lover}

Grunnlaget for telegrafligningen ligger i Kirchhoffs lover, som beskriver hvordan spenning og strøm 
fordeler seg i elektriske kretser. Disse lovene er direkte basert på fysikkens bevaringsprinsipper 
for energi og ladning, og de brukes til å beskrive forholdet mellom elektriske størrelser i ethvert 
kretsnettverk.

\textbf{Kirchhoffs spenningslov (KVL)} sier at summen av alle spenningsendringer rundt en lukket 
sløyfe er lik null. Dette uttrykker energibevaring: den elektriske energien som brukes opp i motstander 
og spoler, må være lik den energien som tilføres av spenningskilder. I telegrafligningen brukes KVL 
til å uttrykke hvordan spenningen endrer seg langs et lite element av ledningen. Spenningsfallet 
langs et element avhenger av den ohmske motstanden og induktansen, og beskrives matematisk som
\begin{equation}
\frac{\partial u}{\partial x} = -R i - L \frac{\partial i}{\partial t}.
\end{equation}

\textbf{Kirchhoffs strømlov (KCL)} sier at summen av alle strømmene som går inn i et knutepunkt, 
er lik summen av de som går ut. Dette er et uttrykk for ladningsbevaring. I telegrafligningen brukes 
KCL til å beskrive hvordan strømmen endrer seg langs ledningen, når en del av strømmen lekker til jord 
gjennom konduktans og kapasitans. Dette gir
\begin{equation}
\frac{\partial i}{\partial x} = -G u - C \frac{\partial u}{\partial t}.
\end{equation}

Ved å kombinere Kirchhoffs lover kan man beskrive hvordan både spenning og strøm endrer seg 
langs ledningen over tid. Dette er utgangspunktet for telegrafligningen, som gir 
en fullstendig matematisk modell for hvordan signaler forplanter seg, dempes og forvrenges når de beveger 
seg gjennom en elektrisk leder.

\subsection{Fourierrekker}

Mange signaler kan brytes ned i sine frekvenskomponenter ved hjelp av Fourier-analyse. Dette er spesielt nyttig i signalbehandling og for å forstå hvordan ulike frekvenser påvirkes av transmisjonslinjen. Ethernet-signaler kan sees på som en sekvens av pulser (bitstrøm), som i praksis kan betraktes som et periodisk pulstog. Dette gjør at Fourierrekker er et naturlig verktøy for å analysere slike signaler i frekvensdomenet, siden pulstoget kan uttrykkes som en sum av sinus- og cosinuskomponenter.\\[1em]
En periodisk funksjon $f(t)$ med periode $T$ kan uttrykkes som en Fourier-rekke:
\begin{equation}
f(t) = a_0 + \sum_{n=1}^{\infty} \left( a_n \cos\left(\frac{2\pi n t}{T}\right) + b_n \sin\left(\frac{2\pi n t}{T}\right) \right)
\end{equation}
med koeffisienter gitt ved:
\begin{align*}
a_0 &= \frac{1}{T} \int_{0}^{T} f(t) \, dt \\
a_n &= \frac{2}{T} \int_{0}^{T} f(t) \cos\left(\frac{2\pi n t}{T}\right) dt \\
b_n &= \frac{2}{T} \int_{0}^{T} f(t) \sin\left(\frac{2\pi n t}{T}\right) dt
\end{align*}
I Kompleks form:
\begin{equation}
f(t) = \sum_{n=-\infty}^{\infty} c_n e^{j \frac{2\pi n t}{T}}, \quad \text{der} \quad c_n = \frac{1}{T} \int_{-T/2}^{T/2} f(t) e^{-j \frac{2\pi n t}{T}} dt, \quad c_0 = \frac{1}{T} \int_{-T/2}^{T/2} f(t) dt
\end{equation}
I dette prosjektet vil vi bruke Fourier-analyse for å dekomponere Ethernet-signaler i sine frekvenskomponenter, slik at vi kan studere hvordan hver komponent påvirkes av transmisjonslinjens egenskaper.
Ved å gjennomføre målinger for forskjellige kabellengder, kan vi sammenligne de målte dataene med de teoretiske prediksjonene basert på Fourier-analyse og telegraflikningen.
Vi får da overføringsfunksjonen for kabelen.
\begin{equation}
    H(f) = \frac{V_{out}(f)}{V_{in}(f)}
\end{equation}
Som beskriver hvordan signalet endres i frekvensdomenet når det passerer gjennom kabelen.
Denne analysen hjelper oss å forstå hvordan ulike frekvenskomponenter dempes og forvrenges, noe som er avgjørende for å forklare begrensningen på kabelens lengde.
Setter vi inn overførsingsfunksjonen i komplekse Fourier-rekken, kan vi modellere hvordan hele signalet endres når det går gjennom kabelen.
\begin{equation}
    f_{out}(t) = \sum_{n=-\infty}^{\infty} H\left(\frac{n}{T}\right) c_n e^{j \frac{2\pi n t}{T}}
\end{equation}
\subsection{Telegraflikningen og transmisjonslinjer}
En transmisjonslinje kan beskrives ved fire per-enhet-lengde parametre:
\begin{itemize}
    \item $R$ \, [$\Omega$/m] \,-- motstand (ledertap),
    \item $L$ \, [H/m] \,-- induktans (magnetisk lagring),
    \item $C$ \, [F/m] \,-- kapasitans (elektrisk lagring),
    \item $G$ \, [S/m] \,-- ledningsevne (lekkasjetap).
\end{itemize}
Ved å kombinere Kirchhoffs lover med disse parameterne utledes telegrafligningen, som beskriver spenning og strøm som funksjon av både posisjon og tid langs kabelen. For spenningen $u(x,t)$ kan den skrives på formen:
\begin{equation}
    u_{tt} + \left(\frac{R}{L} + \frac{G}{C}\right)u_t + \left(\frac{RG}{LC}\right)\,u = \left(\frac{1}{LC}\right) u_{xx}
\end{equation}
\[
    u_{tt} + \left(\alpha + \beta\right)u_t + \alpha\beta\,u = c^2 u_{xx}, \qquad \alpha=\frac{R}{L}, \qquad \beta=\frac{G}{C}, \qquad c = \frac{1}{\sqrt{LC}} ,
\]
der $u(x,t)$ er spenningen langs kabelen og $c$ er utbredelseshastigheten i kabelen.  
Denne ligningen viser at et signal ikke bare forplanter seg med en hastighet bestemt av $L$ og $C$, men også blir dempet og forvrengt på grunn av $R$ og $G$. Resultatet er at høyfrekvente komponenter i signalet svekkes og spres mer enn lave frekvenser, noe som over tid fører til \emph{demping og dispersjon}.  
Ethernet-signaler består ikke av rene sinusbølger, men av pulser med et bredt spekter av frekvenser. Mens Fourier-analyse kan hjelpe oss  å uttrykke signalet som en sum av ulike frekvenskomponenter, gir telegrafligningen oss verktøy til å studere hvordan hver enkelt komponent påvirkes. Dermed kan vi forklare hvorfor en grense på omtrent 100 meter er praktisk: etter denne lengden er tapet og forvrengningen så store at signalet ikke lenger kan tolkes pålitelig av mottakeren.


\subsubsection{R - Resistans}
Resistans per lengde $[\Omega/\mathrm{m}]$ er den ohmske motstanden i lederen. Den skyldes av materialets resistivitet og ledertverrsnitt, og øker med temperatur. Motstanden øker dempingen og reduserer rekkevidden til signalet. I telegrafligningen bidrar R direkte til tidsdemping av både strøm og spenning.

\subsubsection{L - Induktans}
Induktansen per lengde $[\mathrm{H}/\mathrm{m}]$ er magnetisk lagring av energi rundt lederen når strømmen endres, per meter kabel. Sammen med kapasitans vil denne bestemme farten på signalet i kabelen. I linjemodellen står induktansen i serie.

\subsubsection{C - Kapasitans}
Kapasitans per lengde $[\mathrm{F}/\mathrm{m}]$ er elektrisk lagring av energi i feltet mellom leder og referanse, per meter. Sammen med induktans har kapasitans påvirkning på fart. I linjemodellen er dette en gren til jord, den trekker ladning når spenningen endres.

\subsubsection{G - Shunt-ledningsevne (lekkasje)}
$G$ $[\mathrm{S}/\mathrm{m}]$ er shunt-ledningsevnen til jord per lengdeenhet. Den modellerer dielektrisk lekkasje gjennom isolasjonen og står parallelt med kapasitansen $C$ i linjemodellen. Større $G$ gir større demping fordi mer energi tappes i dielektrikumet. Høy fuktighet, aldring eller skadet/dårlig isolasjon øker $G$ (ideell isolasjon gir $G \approx 0$).

\clearpage
\subsubsection{Utledning av telegraflikningen}
Telegraflikningen kan utledes ved å analysere en liten del av transmisjonslinjen, som vist i figuren nedenfor. Vi betrakter et segment av linjen med lengde $\Delta x$, og bruker Kirchhoffs spennings- og strømlov for å sette opp differensialligninger for spenning og strøm.
\begin{figure}[h]
    \centering
    \includegraphics[width=0.6\textwidth]{Media/telegraflinje.png}
    \caption{Elementært segment av en transmisjonslinje med per-enhet-lengde parametere.}
    \label{fig:transmission_line_segment}   
\end{figure}\\
Ved å anvende Kirchhoffs spenningslov på segmentet får vi:
\[
    u(x,t) - u(x+\Delta x,t) = R \Delta x \cdot i(x,t) + L \Delta x \cdot \frac{\partial i(x,t)}{\partial t} .
\]
Ved å dele begge sider av ligningen med $\Delta x$ og lar $\Delta x \to 0$ får vi den første differensialligningen:
\begin{equation}
    \frac{\partial u(x,t)}{\partial x} = -R i(x,t) - L \frac{\partial i(x,t)}{\partial t} . 
\end{equation}
Tilsvarende, ved å bruke Kirchhoffs strømlov, får vi:
\[
    i(x,t) - i(x+\Delta x,t) = -G \Delta x \cdot u(x,t) - C \Delta x \cdot \frac{\partial u(x,t)}{\partial t} .
\]
Igjen, ved å dele begge sider av ligningen med $\Delta x$ og lar $\Delta x \to 0$ får vi den andre differensialligningen:
\[
    \frac{\partial i(x,t)}{\partial x} = -G u(x,t) - C \frac{\partial u(x,t)}{\partial t} .
\]
Ved å derivere den første ligningen med hensyn på $x$ og den andre med hensyn på $t$, og deretter eliminere $i(x,t)$, får vi telegraflikningen for spenningen:
\[
    u_{tt} + \left(\frac{R}{L} + \frac{G}{C}\right)u_t + \left(\frac{RG}{LC}\right)\,u = \left(\frac{1}{LC}\right) u_{xx}
\]
\clearpage
\noindent Dette kan vi skrive som:
\begin{equation}
    u_{tt} + (\alpha + \beta)u_t + \alpha \beta u = c^2 u_{xx}, \qquad c = \frac{1}{\sqrt{LC}} ,
    \label{eq:telegraflikningen}
\end{equation}
der vi har definert:
\[
    \alpha = \frac{R}{L}, \qquad \beta = \frac{G}{C} .
\]
Dette er en dampet bølgeligning som beskriver hvordan spenningen forplanter seg langs transmisjonslinjen, med demping og forvrengning bestemt av parametrene $R$, $L$, $C$, og $G$.



\subsubsection{Propagerende løsning og overføringsfunksjon}

I forrige del utledet vi telegraflikningen for spenning langs en kabel uttrykt ved de per-lengde-parametrene $R, L, G, C$ som beskriver hvordan spenningen varierer både i tid og rom \eqref{eq:telegraflikningen}. Denne likningen er en såkalt bølgelikning med demping. Vi forventer derfor at løsningene skal ha bølgekarakter som oscillerer i tid og samtidig forplanter seg langs kabelen. For å fange dette opp antar vi en løsning på formen
\begin{equation}
V(x,t) = \Re\!\left\{ U e^{j\omega t - \gamma(\omega)x} \right\}.
\label{eq:harm-ansatz}
\end{equation}
Her har vi valgt tre elementer:
\begin{itemize}
    \item \textbf{Tidsdelen $e^{j\omega t}$:} beskriver en harmonisk oscillasjon i tid. At vi bruker den 
    komplekse eksponentialen i stedet for $\cos(\omega t)$ eller $\sin(\omega t)$ skyldes at dette 
    gjør beregningene enklere. Til slutt tar vi realdelen for å hente ut det fysiske signalet.
    \item \textbf{Romdelen $e^{-\gamma x}$:} beskriver hvordan signalet utvikler seg langs kabelens lengde. 
    Hvis $\gamma$ er reell, betyr dette ren demping. Hvis $\gamma$ er imaginær, betyr det ren faseforskyvning. 
    I praksis er $\gamma$ kompleks, slik at vi får begge deler.
    \item \textbf{Kompleks amplitude $U$:} setter størrelsen og fasen på bølgen.\\
\end{itemize}
Dette kalles en \emph{harmonisk ansats}. Den er valgt fordi vi vet fra Fourier-teori at alle signaler 
kan bygges opp som en sum av slike harmoniske komponenter. Når vi kjenner kabelens respons på en frekvens, 
kan vi utvide til vilkårlige signaler. Setter vi \eqref{eq:harm-ansatz} inn i telegraflikningen \eqref{eq:telegraflikningen}:
\[    
    V_{tt} + (\alpha + \beta)V_{t} + \alpha \beta V = c^2 V_{xx}.
\]
Ser vi at den tidslige avledningen bare gir faktorer av $j\omega$:
\[
    V_{t} = j\omega V, \quad V_{tt} = -\omega^2 V,
\]
og den romlige avledningen gir faktorer av $\gamma$:
\[
    V_{x} = -\gamma V, \quad V_{xx} = \gamma^2 V.
\]
Som gir oss:
\[    
-\omega^2 V + j\omega(\alpha+\beta)V + \alpha \beta V = c^2 \gamma^2 V.
\]
Vi kan forkorte med $V$ (som er forskjellig fra null) og får:
\[
c^2 \gamma^2(\omega) = \alpha \beta - \omega^2 + j\omega(\alpha+\beta).
\]
\clearpage
Dette er en såkalt \emph{dispersjonsrelasjon}, som forteller hvilke kombinasjoner av 
frekvens $\omega$ og bølgetall $\gamma$ som er mulige. Vi kan løse for $\gamma$ og får
\[
\gamma(\omega) = \frac{1}{c}\sqrt{\alpha\beta - \omega^2 + j\omega(\alpha+\beta)}.
\]
Setter vi inn definisjonene på $\alpha,\beta$ og $c$, kan dette skrives på den standardformen
som brukes i transmisjonslinjeteori:
\begin{equation}
\;\gamma(\omega) = \sqrt{(R+j\omega L)(G+j\omega C)}\;
\label{eq:prop-konstant}
\end{equation}
Dette uttrykket viser tydelig hvordan både ledertap ($R$) og dielektriske tap ($G$) bidrar 
til at signalet dempes, mens $L$ og $C$ bestemmer hastigheten til bølgen. Vi kan også uttrykke $\gamma$
i form av sin reelle og imaginære del:
\[
\gamma(\omega) = \alpha_p(\omega) + j\beta_p(\omega).
\]
Her har $\alpha_p(\omega)$ enheten Np/m (neper per meter) og forteller hvor raskt amplituden 
avtar med lengden, mens $\beta_p(\omega)$ har enheten rad/m og forteller hvor raskt fasen 
øker langs kabelen. Fra $\beta_p$ kan vi utlede bølgens fasehastighet:
\[v_p = \frac{\omega}{\beta_p}\]
og dermed hvor stor tidsforsinkelse signalet får etter en viss lengde.\\[1em]
Hvis vi nå ser på en kabel av lengde $l$, vil en harmonisk komponent ved frekvens $\omega$ 
endres med faktoren
\begin{equation}
H(\omega,l) = e^{-\gamma(\omega)l} = e^{-\alpha_p(\omega)l}\,e^{-j\beta_p(\omega)l}
\end{equation}
Dette er kabelens \emph{overføringsfunksjon}. 
Den kan forstås slik:
\begin{itemize}
    \item $e^{-\alpha_p l}$: amplituden reduseres eksponentielt med lengden, 
    \item $e^{-j\beta_p l}$: signalet forskyves i fase, tilsvarende en tidsforsinkelse.\\[1em]
\end{itemize}
Så langt har vi bare sett på en harmonisk bølge. Men i praksis er alle signaler satt sammen av mange slike bølger. 
Et vilkårlig innsignal kan skrives som en Fourier-rekke:
\[
f_{\text{in}}(t) = \sum_n c_n e^{j\omega_n t}.
\]
etter kabelen vil hver komponent være påvirket av $H(\omega_n,l)$, og vi får
\[
f_{\text{out}}(t) = \sum_n H(\omega_n,l)\,c_n e^{j\omega_n t}.
\]
Dermed virker kabelen som et frekvensavhengig filter: lave frekvenser slipper nesten uendret gjennom, 
mens høye frekvenser dempes kraftig. For firkantsignaler betyr dette at de skarpe kantene, 
som består av mange høye frekvenser, gradvis blir avrundet etter hvert som lengden øker. 
Det er nettopp dette vi kan observere i laboratoriet når vi sender firkantpulser gjennom en Ethernet-kabel.

\clearpage
\subsection{Demping (amplituderespons)}
Demping beskrives av realdelen av propagasjonskonstanten:
\[ 
  \alpha_p(\omega)=\Re\{\gamma(\omega)\}
\]
Den sier noe om hvor stor faktor hver frekvenskomponent dempes i amplitude. Denne faktoren er definert som størrelsen av overføringsfunksjonen:
\[
|H(\omega,l)|=e^{-\alpha_p(\omega)\,l}, \qquad |H(\omega,l)|_{\mathrm{dB}} = 20\log_{10}|H| = -8.686\,\alpha_p(\omega)\,l
\]
Når $c_n$ er Fourier-koeffisienten til n-te harmoniske komponent, er amplituden til denne komponenten etter kabelen gitt ved:
\[
|c_n^{out}| = |H(\omega_n,l)| \cdot |c _n| = e^{-\alpha_p(\omega_n) l} |c_n|
\]
Hvorfor signalet dempes kan forklares med to hovedmekanismer:\\
\begin{itemize}[leftmargin=2.8em,style=nextline]
  \item \textbf{Skinneffekt}: Når frekvensen øker, presses strømmen mot overflaten av lederen. Det effektive tverrsnittet blir mindre, og den elektriske motstanden per lengde øker. For runde ledere (og godt approksimert i praksis) gjelder:
  \[
  R(\omega)\ \propto\ \sqrt{\omega}
  \]
  Ut fra propagasjonskonstanten \eqref{eq:prop-konstant} følger at $\alpha_p(\omega)$ vokser omtrent som $\sqrt{\omega}$.\\
  \item \textbf{Dielektriske tap}: Et virkelig dielektrikum er ikke tapsfritt: polarisasjonen henger litt etter feltet (faseforsinkelse), noe som kan modelleres med \emph{loss tangent} \(\tan\delta\). For små tap får vi den nyttige tilnærmingen
  \[
  G(\omega)\ =\ \omega\,C\,\tan\delta \quad \Rightarrow \quad G(\omega)\ \propto\ \omega
  \]
  slik at tapet øker lineært med frekvens. Dette bidrar også til \(\alpha_p(\omega)\) og dermed til økende demping med \(\omega\).\\
\end{itemize}
\clearpage

Dette gir da direkte utslag i overføringsfunksjonen. Siden både \(R(\omega)\) og \(G(\omega)\) øker med frekvens, dempes høyfrekvente komponenter mest.
I tid betyr det sløvere kanter og “droop” på platået (vertikal lukking i øyediagram) siden høye frekvenser som trengs for å gjenskape skarpe kanter, er mest dempet.
\begin{figure}[h]
    \centering
    \includegraphics[width=0.9\textwidth]{Media/frekvensdoment_tidsdoment.png}
    \caption{Tids- og frekvensdomene av original og modifisert bølgeform. Pulsformen bestemmes av forholdet mellom frekvenskomponentene; er amplituder og relative faser like, blir tidskurven uendret.}
    \label{fig:eyediagram}
\end{figure}
\subsection{Dispersjon (fase, gruppeforsinkelse og gruppehastighet)}
Dispersjon beskrives av den imaginære delen av propagasjonskonstanten (fasekonstanten):
\[
    \beta_p(\omega)=\Im\{\gamma(\omega)\}
\]
Den sier noe om hvor mye hver frekvenskomponent forskyves i fase.
Fasen til kabelen er gitt ved:
\[
\varphi(\omega,l) = -\beta_p(\omega)\,l
\]
Fra fasen kan vi finne \emph{gruppeforsinkelse} og \emph{gruppehastighet}.
\emph{Gruppeforsinkelse} er definert som:
\[
\tau_g(\omega,l) = \varphi'(\omega,l) = -\frac{d}{d\omega}\varphi(\omega,l) = l\,\frac{d\beta_p(\omega)}{d\omega}
\]
Den sier noe om hvor mye en smal båndbredde rundt frekvensen \(\omega\) forsinkes i tid etter å ha passert kabelen. \emph{Gruppehastighet} er definert som:
\[
v_g(\omega) = \frac{1}{\tau_{g}'(\omega)} = \left(\frac{d\beta_p(\omega)}{d\omega}\right)^{-1}, \quad hvor \quad \tau_g'(\omega) = \frac{\tau_g(\omega,l)}{l} = \frac{d\beta_p(\omega)}{d\omega} \quad [\mathrm{s/m}]
\]
\clearpage
\noindent Den sier noe om hvor raskt en smal båndbredde rundt frekvensen \(\omega\) forplanter seg i kabelen. Det vi ser her er at \(\tau_g\) og \(v_g\) avhenger av frekvensen og det er netteopp dette som kalles \emph{dispersjon}. Vi kan tolke dette slik at ulike frekvenskomponenter av et signal forplanter seg med ulik hastighet. Dette får direkte konsekvenser for pulsform og øyediagram.\\
\begin{itemize}[leftmargin=2.8em,style=nextline]
  \item \textbf{Ingen dispersjon:} Hvis \(\frac{d\beta_p}{d\omega} \) er konstant (lineær fase), får alle frekvenskomponenter samme tidsforsinkelse \(\Rightarrow\) ingen pulsbredding (ingen horisontal øyelukking).\\
  \item \textbf{Dispersjon:} Når \(\frac{d\beta_p}{d\omega}\) varierer med frekvens (ikke-lineær fase), får ulike frekvenskomponenter ulik tidsforsinkelse \(\Rightarrow\) pulsbredding (horisontal øyelukking). I lavtap-tilfellet er ofte 
  \[
    \beta_p(\omega)\approx \omega\sqrt{LC}\quad,\quad slik\quad at \quad \tau_{g}'(\omega)=\frac{d\beta_p}{d\omega}\approx \sqrt{LC}\quad,\quad v_g(\omega)\approx \frac{1}{\sqrt{LC}}
  \]
  (nær konstant): da blir formen mest påvirket av demping, ikke fase-dispersjon.\\
\end{itemize}
\textbf{Spesialtilfelle (forvrengningsfri linje):}\\
Heaviside-betingelsen:
\[
    \alpha = \beta \quad \Rightarrow \quad \frac{R}{L}=\frac{G}{C}
\]
gir \(\beta_p(\omega)\) lineær i \(\omega\)
(konstant \(\tau_g'\)) og \(\alpha_p(\omega)\) frekvensuavhengig. Da fås ingen dispersiv forvrengning (kun lik demping av alle frekvenser).


\subsection{Kobling til pulstog}
Et periodisk innsignal med Fourierkoeffisienter \(c_n\) filtreres som
\[
f_{\text{out}}(t)=\Re\!\left\{\sum_{n=-N}^{N} c_n\,H(\omega_n,l)\,e^{j\omega_n t}\right\}.
\]
Demping (via \(\alpha_p\)) fjerner spesielt de høye leddene \(\Rightarrow\) langsommere kanter.\\
Dispersjon via: 
\[\frac{d\beta_p}{d\omega}\] 
tidsforskyver leddene relativt \(\Rightarrow\) pulsbredding.


\subsection{Oppsummering og kobling til prosjektet}

Teorien over forklarer hvorfor signaler i lange kabler dempes og forvrenges, og gir oss verktøyene til å analysere dette både analytisk og numerisk. Dette er avgjørende for å forstå hvorfor det er en anbefalt grense for Ethernet-kabler. I det videre arbeidet skal vi først gjennomføre praktiske målinger, og deretter \textcolor{red}{numeriske simuleringer} (kanskje), slik at begge kan sammenlignes med de teoretiske resultatene for å se hvordan modellen stemmer med virkeligheten.
