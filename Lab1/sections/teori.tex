\section{Teori}
IEEE \parencite{ieeeGuide} gir en god oversikt over grunnleggende kretsteori. 
I denne rapporten vil vi fokusere på de mest relevante teoriene for oppgaven. 
APA\cite{apaGuide} gir en god oversikt over hvordan dette også kan brukes

\subsection{Resistor}
\dots

\begin{equation}
V = R \cdot I
\end{equation}

\subsection{Telegrafligningen}
\begin{equation}
u_{tt} + (\alpha + \beta) u_t + \alpha \beta u = c^2 u_{xx}.
\end{equation}

\subsection{Utledning av telegrafligningen}

Vi modellerer en elektrisk kabel ved å betrakte et lite element med lengde $\Delta x$. Elementet beskrives ved fire parametre per lengdeenhet:  
\begin{itemize}
    \item Resistans $R$ [$\Omega$/m]  
    \item Induktans $L$ [H/m]  
    \item Kapasitans $C$ [F/m]  
    \item Konduktans $G$ [S/m]  
\end{itemize}

Disse størrelsene representerer henholdsvis ohmsk tap, magnetisk energi, elektrisk lagring av ladning og lekkasje til jord.  

\paragraph{Spenningslikning.}  
La $u(x,t)$ være spenningen og $i(x,t)$ strømmen. Ifølge Kirchhoffs spenningslov er spenningsfallet langs et element
\begin{equation}
\frac{\partial u}{\partial x} = -R i - L \frac{\partial i}{\partial t}.
\end{equation}

\paragraph{Strømlikning.}  
Ifølge Kirchhoffs strømlov lekker strøm ut gjennom konduktansen og kapasitoren:
\begin{equation}
\frac{\partial i}{\partial x} = -G u - C \frac{\partial u}{\partial t}.
\end{equation}

\paragraph{Kombinasjon.}  
Vi deriverer spenningslikningen med hensyn på $x$ og setter inn uttrykket for $\partial i/\partial x$:
\begin{align}
\frac{\partial^2 u}{\partial x^2} 
&= -R \frac{\partial i}{\partial x} - L \frac{\partial^2 i}{\partial t \partial x} \\
&= R(G u + C u_t) + L(G u_t + C u_{tt}).
\end{align}

\paragraph{Omforming.}  
Samler vi leddene får vi:
\begin{equation}
u_{xx} = (RC) u_t + (LC) u_{tt} + (RG) u + (LG) u_t.
\end{equation}
Dividerer vi med $LC$:
\begin{equation}
u_{tt} + \frac{R}{L} u_t + \frac{G}{C} u_t + \frac{RG}{LC} u = \frac{1}{LC} u_{xx}.
\end{equation}

\paragraph{Standardform.}  
Vi definerer
\begin{equation}
\alpha = \frac{R}{2L}, \quad 
\beta = \frac{G}{2C}, \quad 
c = \frac{1}{\sqrt{LC}}.
\end{equation}
Dermed får vi den standardiserte telegrafligningen:
\begin{equation}
u_{tt} + (\alpha + \beta) u_t + \alpha \beta u = c^2 u_{xx}.
\end{equation}

Denne partielldifferensialligningen beskriver spenningen i kabelen, og et tilsvarende uttrykk kan utledes for strømmen.
