
\section{Teori}

    \textbf{Motivasjon og sammenheng:} For å forstå hvorfor det er en grense på 100 meter for Ethernet-kabler, og hvordan signaler dempes og forvrenges i slike kabler, må vi bruke telegraflikningen. Denne beskriver hvordan elektriske signaler forplanter seg i transmisjonslinjer, og gir oss verktøyene til å analysere tap, dispersjon.

\subsection{Fourierrekker}

Mange signaler kan brytes ned i sine frekvenskomponenter ved hjelp av Fourier-analyse. Dette er spesielt nyttig i signalbehandling og for å forstå hvordan ulike frekvenser påvirkes av transmisjonslinjen. Vi bruker Fourier-rekker for endelige intervaller (Dirichlet/Neumann). Ethernet-signaler kan sees på som en sekvens av pulser (bitstrøm), som i praksis kan betraktes som et periodisk pulstog. Dette gjør at Fourierrekker er et naturlig verktøy for å analysere slike signaler i frekvensdomenet, siden pulstoget kan uttrykkes som en sum av sinus- og cosinuskomponenter.\\[1em]
En periodisk funksjon $f(t)$ med periode $T$ kan uttrykkes som en Fourier-rekke:
\begin{equation}
f(t) = a_0 + \sum_{n=1}^{\infty} \left( a_n \cos\left(\frac{2\pi n t}{T}\right) + b_n \sin\left(\frac{2\pi n t}{T}\right) \right)
\end{equation}
med koeffisienter gitt ved:
\begin{align*}
a_0 &= \frac{1}{T} \int_{0}^{T} f(t) \, dt \\
a_n &= \frac{2}{T} \int_{0}^{T} f(t) \cos\left(\frac{2\pi n t}{T}\right) dt \\
b_n &= \frac{2}{T} \int_{0}^{T} f(t) \sin\left(\frac{2\pi n t}{T}\right) dt
\end{align*}
I Komplekse form:
\begin{equation}
f(t) = \sum_{n=-\infty}^{\infty} c_n e^{i \frac{2\pi n t}{T}}, \quad \text{der} \quad c_n = \frac{1}{T} \int_{0}^{T} f(t) e^{-i \frac{2\pi n t}{T}} dt
\end{equation}
I dette prosjektet vil vi bruke Fourier-analyse for å dekomponere Ethernet-signaler i sine frekvenskomponenter, slik at vi kan studere hvordan hver komponent påvirkes av transmisjonslinjens egenskaper.
Ved å gjennomføre målinger for forskjellige kabellengder, kan vi sammenligne de målte dataene med de teoretiske prediksjonene basert på Fourier-analyse og telegraflikningen.
Vi får da overføringsfunksjonen for kabelen.
\begin{equation}
    H(f) = \frac{V_{out}(f)}{V_{in}(f)}
\end{equation}
Som beskriver hvordan signalet endres i frekvensdomenet når det passerer gjennom kabelen.
Denne analysen hjelper oss å forstå hvordan ulike frekvenskomponenter dempes og forvrenges, noe som er avgjørende for å forklare begrensningen på kabelens lengde.
Setter vi inn overførsingsfunksjonen i komplekse Fourier-rekken, kan vi modellere hvordan hele signalet endres når det går gjennom kabelen.
\begin{equation}
    f_{out}(t) = \sum_{n=-\infty}^{\infty} H\left(\frac{n}{T}\right) c_n e^{i \frac{2\pi n t}{T}}
\end{equation}
\subsection{Telegraflikningen og transmisjonslinjer}
En transmisjonslinje kan beskrives ved fire per-enhet-lengde parametre:
\begin{itemize}
    \item $R$ \, [$\Omega$/m] \,-- motstand (ledertap),
    \item $L$ \, [H/m] \,-- induktans (magnetisk lagring),
    \item $C$ \, [F/m] \,-- kapasitans (elektrisk lagring),
    \item $G$ \, [S/m] \,-- ledningsevne (lekkasjetap).
\end{itemize}
Ved å kombinere Kirchhoffs lover med disse parameterne utledes telegrafligningen, som beskriver spenning og strøm som funksjon av både posisjon og tid langs kabelen. For spenningen $u(x,t)$ kan den skrives på formen:
\begin{equation}
    u_{tt} + (RG + LC)u_t + RG\,u = c^2 u_{xx}, \qquad c = \frac{1}{\sqrt{LC}} ,
\end{equation}
der $u(x,t)$ er spenningen langs kabelen og $c$ er utbredelseshastigheten i kabelen.  
Denne ligningen viser at et signal ikke bare forplanter seg med en hastighet bestemt av $L$ og $C$, men også blir dempet og forvrengt på grunn av $R$ og $G$. Resultatet er at høyfrekvente komponenter i signalet svekkes og spres mer enn lave frekvenser, noe som over tid fører til \emph{demping og dispersjon}.  
Ethernet-signaler består ikke av rene sinusbølger, men av pulser med et bredt spekter av frekvenser. Med Fourier-analyse kan signalet uttrykkes som en sum av slike frekvenskomponenter, og telegrafligningen gir oss verktøy til å studere hvordan hver enkelt komponent påvirkes. Dermed kan vi forklare hvorfor en grense på omtrent 100 meter er praktisk: etter denne lengden er tapet og forvrengningen så store at signalet ikke lenger kan tolkes pålitelig av mottakeren.

\subsubsection{Utledning av telegraflikningen}
Telegraflikningen kan utledes ved å analysere en liten del av transmisjonslinjen, som vist i figuren nedenfor. Vi betrakter et segment av linjen med lengde $\Delta x$, og bruker Kirchhoffs spennings- og strømlov for å sette opp differensialligninger for spenning og strøm.
\begin{figure}[h]
    \centering
    \includegraphics[width=0.6\textwidth]{Media/telegraflinje.png}
    \caption{Elementær segment av en transmisjonslinje med per-enhet-lengde parametere.}
    \label{fig:transmission_line_segment}   
\end{figure}
Ved å anvende Kirchhoffs spenningslov på segmentet får vi:
\begin{equation}
    u(x,t) - u(x+\Delta x,t) = R \Delta x \cdot i(x,t) + L \Delta x \cdot \frac{\partial i(x,t)}{\partial t} .
\end{equation}
Ved å la $\Delta x \to 0$ får vi den første differensialligningen:
\begin{equation}
    \frac{\partial u(x,t)}{\partial x} = -R i(x,t) - L \frac{\partial i(x,t)}{\partial t} . 
\end{equation}
Tilsvarende, ved å bruke Kirchhoffs strømlov, får vi:
\begin{equation}
    i(x,t) - i(x+\Delta x,t) = -G \Delta x \cdot u(x,t) - C \Delta x \cdot \frac{\partial u(x,t)}{\partial t} .
\end{equation}
Igjen, ved å la $\Delta x \to 0$ får vi den andre differensialligningen:
\begin{equation}
    \frac{\partial i(x,t)}{\partial x} = -G u(x,t) - C \frac{\partial u(x,t)}{\partial t} .
\end{equation}
Ved å derivere den første ligningen med hensyn på $x$ og den andre med hensyn på $t$, og deretter eliminere $i(x,t)$, får vi telegraflikningen for spenningen:
\begin{equation}
    u_{tt} + (RG + LC)u_t + RG\,u = c^2 u_{xx}, \qquad c = \frac{1}{\sqrt{LC}} .
\end{equation}
Dette kan vi skrive som:
\begin{equation}
    u_{tt} + (\alpha + \beta)u_t + \alpha \beta u = c^2 u_{xx}, \qquad c = \frac{1}{\sqrt{LC}} ,
\end{equation}
der vi har definert:
\begin{equation}
    \alpha = \frac{R}{L}, \qquad \beta = \frac{G}{C} .
\end{equation}
Dette er en dampet bølgeligning som beskriver hvordan spenningen forplanter seg langs transmisjonslinjen, med demping og forvrengning bestemt av parametrene $R$, $L$, $C$, og $G$.

\subsection{Metoder og praktisk relevans}

(a) \textit{Fourierrekker:} Analytisk på $[0,L]$ (m/Dirichlet) for å vise $\alpha=\beta$-tilfellet.\\
\dots
\clearpage
\subsection{Dispersjon}

Dispersjon betyr at ulike frekvenskomponenter i en bølge forplanter seg med forskjellig hastighet. 
I telegrafligningen oppstår dette fordi bølgefarten $v_n$ avhenger av frekvensen og parameterne 
$R$, $L$, $C$ og $G$. 

Konsekvensen er at et signal som består av mange frekvenser (for eksempel en skarp puls) gradvis 
forvrenges når det forplanter seg. Dette er en sentral utfordring i signaloverføring: skarpe signaler 
blir bredere og mister form over tid. 
\dots

\subsection{Oppsummering og kobling til prosjektet}

Teorien over forklarer hvorfor signaler i lange kabler dempes og forvrenges, og gir oss verktøyene til å analysere dette både analytisk og numerisk. Dette er avgjørende for å forstå hvorfor det er en 100 m-grense for Ethernet-kabler. I det videre arbeidet skal vi først gjennomføre praktiske målinger, og deretter \textcolor{red}{numeriske simuleringer} (kanskje), slik at begge kan sammenlignes med de teoretiske resultatene for å se hvordan modellen stemmer med virkeligheten.
