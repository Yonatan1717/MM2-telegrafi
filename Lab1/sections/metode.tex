\section{Metode}

\subsection{Mål og hypoteser}
\textbf{Formål:} analysere hvordan en transmisjonslinje (koaksialkabel) påvirker et pulstog i tids- og frekvensdomenet, og validere en RLCG-modell basert på telegrafligningen, numerisk simulering og laboratoriemålinger. Den estimerte overføringsfunksjonen predikerer målt demping og fase innenfor forhåndsdefinerte akseptgrenser. Overføringsfunksjonen:

\[
H(\omega) = e^{-\gamma(\omega)\ell} = e^{-\sqrt{(R+j\omega L)(G+j\omega C)}\,\ell}
\]

\subsection{Oversikt}
Metoden vi bruker i denne rapporten består av to hoveddeler: numerisk simulering og laboratoriestudie. Den numeriske simuleringen innebærer å modellere transmisjonslinjen ved hjelp av telegrafligningen. Vi bruker Python for å implementere denne modellen og generere simuleringsdata. I laboratoriestudien gjennomfører vi praktiske målinger av hvordan et pulstog påvirkes når det sendes gjennom en koaksialkabel av forskjellige lengder. Vi bruker en funksjonsgenerator til å generere pulstoget og et oscilloskop for å måle inngangs- og utgangssignaler.

\subsection{Numerisk simulering og modellering}
For å simulere telegrafligningen numerisk brukte vi Python med relevante biblioteker for numerisk beregning. Koden er vedlagt i vedlegg \ref{appendix:code}.

\paragraph{Modellantakelser.}
Vi modellerer kabelen som en lineær, tidsinvariant linje med fordelte parametre $R(f),L,C,G(f)$ per meter. Merk at $R$ og $G$ er frekvensavhengige for å modellere henholdsvis skinneffekt og dielektriske tap. Vi antar at kabelen er terminert med en last som matcher karakteristisk impedans, slik at vi kan se bort fra refleksjoner, i tillegg er verdier for $L$ og $C$ hentet fra datablad. Vi bruker funksjonen:
\[
\gamma(\omega)=\sqrt{(R+j\omega L)(G+j\omega C)},\qquad
H(\omega)=e^{-\gamma(\omega)\,\ell}
\]

\noindent Verdier for RGLC vil være følgende: \\
$R(f) = 0.17 \cdot \sqrt{\frac{f}{67 \cdot 10^3}} \qquad [\Omega / \mathrm{m}]$. \\
$L = 525 \mathrm{nH/m}$ \\
$C = 52 \mathrm{pF/m}$ \\
$G(f) = 2\pi f C \tan(\delta)$, \qquad $\tan(\delta) = 0.002$, \qquad [S/m]

\paragraph{Simuleringens modellering.}
Vi modellerer 5 forskjellige situasjoner for å forstå hvordan forskjellige medier påvirker signalet som propagerer gjennom. Lengdene på mediet er 10m, 50m, 100m og 1000m. Simuleringen tar utgangspunkt i formelen:
\[
\gamma (f) = \alpha(f) + j\beta(f), \qquad \gamma(f) = \gamma(\omega)|_{\omega = 2\pi f}
\]

\noindent Vi simulerer følgende modeller:
\begin{itemize}
    \item \textbf{Kun demping:} Ignorer faseendring ved å sette $\beta = 0$.
    \item \textbf{Kun faseendring:} Ignorer demping ved å sette $\alpha = 0$.
    \item \textbf{Ideell, tapsfri og ikke-dispersiv modell:} Mediet er ideelt, uten demping og uten dispersjon.
    \item \textbf{Full modell:} Inkluderer både demping og faseendring. Simuleringen tilnærmer seg virkeligheten.
    \item \textbf{Waterfall plott:} Simulerer hvordan forskjellige frekvenskomponenter påvirker tidsdomenesignalet.
\end{itemize}

\subsubsection{Inngangssignal og representasjon}
Vi modellerer pulstoget som en 50\% duty-cycle firkantpuls med periodetid $T = 2 \mu s$ og grunnvinkelhastighet $\omega_0 = 2\pi / T$. Siden duty-cycle er 50\% består spekteret av kun oddetalls harmoniske. Tidsfunksjonen representeres med Fourier-rekken.
\begin{equation}
    V_{\mathrm{in}}(t) = \frac{5}{2} + \sum_{\substack{n=-N_{\mathrm{harm}}\\ n\ \text{oddetall}}}^{N_{\mathrm{harm}}}\frac{5}{jn\pi} e^{jn\omega_0 t}
\end{equation} 

\subsection{Laboratoriestudie}
\subsubsection{Gjennomførelse} Dette er vårt praktiske forsøk for å studere og analysere telegrafligningen i virkeligheten. Vi bruker en funksjonsgenerator til å generere et firkant-pulstog som sendes gjennom en koaksialkabel av kjente forskjellige lengder. På mottakersiden bruker vi et oscilloskop for å sammenlikne data med en kort, direktekoblet koaksialkabel (referanse) og gjennom en lengre koaksialkabel.

\subsubsection{Hensikt} Hensikten med laberatorieforsøket er å undersøke hvordan signalet endres når det overføres gjennom en transmisjonslinje over forskjellige lengder. Ved høyere frekvenser forventes amplitudetap, faseforskyvning og forvrengning av signalet, som vi ønsker å kvantifisere og analysere.

\subsubsection{Utstyr og oppsett}
Utstyret i forsøket er følgende:
\begin{itemize}
    \item \textbf{Funksjonsgenerator.} Genererer firkant-pulstog med justerbar frekvens og amplitude. Firkant-pulstoget ble konfigurert til periodetid $T = 2 \mu s$ (frekvens $f_\mathrm{ref} = 500 \mathrm{kHz}$) med 50\% duty-cycle og amplitude på $2.5V_{pp}$. Startfasen ble satt til $0\si{\degree}$ for å sikre at signalet starter ved null.
    \item \textbf{Oscilloskop.} FFT-funksjonalitet for frekvensanalyse.
    \item \textbf{Kabler.} Koaksialkabel med lengder på 5m og 30m.
\end{itemize}

\subsubsection{Måleprosedyrer}
Måleprosedyrene inkluderer kalibrering av funksjonsgenerator og oscilloskop, oppsett av firkant-pulstog, og måling av inngangs- og utgangssignaler for hver kabellengde. På oscilloskopet målte vi $V_{in}$ og $V_{out}$ for hver kabellengde, og registrerte dataene for videre analyse. Her benyttet vi oss også av FFT (Fast Fourier Transform) for å analysere frekvensspekteret til signalene. Vi gjennomfører gjentatte målinger for å vurdere konsistens og usikkerhet.

\subsubsection{FFT - Fast Fourier Transform}
Vi benytter Fast Fourier Transform (FFT) på oscilloskopet for å konvertere tidsserier til frekvensdomenet og dermed studere systemets frekvensrespons. FFT er en effektiv algoritme for å beregne den diskrete Fourier-transformasjonen (DFT) og brukes blant annet i oscilloskopet for å vise frekvensspekteret. I motsetning til direkte DFT-beregning med kjøretid $O(N^2)$, har FFT kjøretid $O(N\log N)$, noe som muliggjør rask og effektiv analyse av store datasett og sanntidsmålinger.
