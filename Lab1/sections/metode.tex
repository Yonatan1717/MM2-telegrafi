\section{Metode}

\subsection{Oversikt}
Vi kombinerer numerisk modellering av transmisjonslinjen med laboratoriemålinger, og sammenlikner frekvens- og tidsdomene-respons for å validere modellen. Vi simulerer i Python og utfører målinger med funksjonsgenerator og oscilloskop. Data presenteres under modellering og resultater. For å analysere og simulere telegrafligningen og dens effekter på signaloverføring i transmisjonslinjer, benyttet vi følgende metoder: \\


\begin{itemize}
    \item \textbf{Numerisk simulering:} \textcolor{red}{Vi implementerte telegrafligningen i et numerisk simuleringsverktøy (f.eks. MATLAB eller Python) for å modellere signaloverføring over en gitt lengde av transmisjonslinjen. Dette inkluderte å simulere impulsresponsen og overføringsfunksjonen.}
    \item \textbf{Parameterstudie:} \textcolor{red}{Vi varierte parametrene R, L, G, og C for å observere deres individuelle og kollektive effekter på signaldemping og forvrengning. Dette hjalp oss med å identifisere hvilke faktorer som har størst innvirkning på signalintegriteten.}
    \item \textbf{Labaratoriestudie:}\textcolor{red}{ Vi gjennomførte eksperimenter med faktiske kabler og måleutstyr for å validere våre simuleringer. Dette inkluderte måling av signalrespons ved forskjellige frekvenser og sammenligning med teoretiske prediksjoner.}
\end{itemize}

\subsection{Numerisk modell}
\paragraph{Modellantakelser.}
Vi modellerer kabelen som en lineær, tidsinvariant linje med fordelte parametre $R(f),L,C,G(f)$ per meter og bruker
\[
\gamma(\omega)=\sqrt{(R+j\omega L)(G+j\omega C)},\qquad
H(\omega)=e^{-\gamma(\omega)\,\ell}.
\]

\paragraph{Frekvensavhengigheter.}
Resistans skaleres med skinneffekt $R(f)$ (for $f>0$), og dielektriske tap modelleres ved
$G(f)$. Disse parameterne hentes/justeres fra datablad og brukes som startverdier i tilpasning.

\paragraph{Inngangssignal og representasjon.}
Vi modellerer pulstoget som en 50\% duty-cycle firkantpuls med periodetid $T=1/f_\mathrm{ref} = 100 \mathrm{ns}$ og grunnvinkelhastighet $\omega_0 = 2\pi / T$. Siden duty-cycle er 50\% består spekteret av kun oddetalls harmoniske. Tidsfunksjonen representeres med Fourier-rekken.
\begin{equation}
    V_{\mathrm{in}}(t) = \frac{5}{2} + \sum_{n=1,3,5,...}^{N_{\mathrm{harm}}}\frac{5}{jn\pi} e^{jn\omega_0 t}
\end{equation} 

\paragraph{Implementasjonsdetaljer.}
Python $3.x$ med bibliotekene NumPy og Matplotlib. Kode og versjoner vedlegges. Vi dokumenterer alle faste parametre ($R_0,L,C,\tan\delta,f_\mathrm{ref}$) og kabel-lengder.



\subsection{Laboratoriestudie}
\paragraph{Gjennomførelse}
Dette er vårt praktiske forsøk for å studere og analysere telegrafligningen i virkeligheten. Vi bruker en funksjonsgenerator (modell?) til å generere et firkant-pulstog med 50\% duty-cycle og frekvens $f_\mathrm{ref} = 10 \mathrm{MHz}$, som sendes gjennom en koaksialkabel av kjent lengde. På mottakersiden bruker vi et oscilloskop for å sammenlikne data av en kort, direktekobling (referanse) og gjennom en lengre koaksialkabel.
\paragraph{Hensikt.} Hensikten med laberatorieforsøket er å undersøke hvordan signalets attributer som amplitude og form endres når det overføres gjennom en transmisjonslinje, og å validere den numeriske modellen ved å sammenligne målte data med simuleringsresultater og teoretiske prediksjoner basert på telegrafligningen. Ved høyere frekvenser forventes amplitudetap, faseforskyvning og forvrengning av signalet, som vi ønsker å kvantifisere og analysere.

\paragraph{Utstyr og oppsett.}
Utstyret i forsøket er følgende:
\begin{itemize}
    \item \textbf{Funksjonsgenerator:} Modell og spesifikasjoner (frekvensområde, oppløsning, nøyaktighet).
    \item \textbf{Oscilloskop:} Modell og spesifikasjoner (båndbredde, samplingsrate, oppløsning).
    \item \textbf{Kabler:} Type (koaksial), lengder (f.eks. 1m, 10m, 50m), og relevante parametre ($R_0,L,C,\tan\delta$ fra datablad).
    \item \textbf{Miljø:} Temperatur og andre forhold som kan påvirke målingene.
\end{itemize}

\paragraph{Måleprosedyrer.}
Måleprosedyrene inkluderer:
\begin{itemize}
    \item Kalibrering av funksjonsgenerator og oscilloskop.
    \item Oppsett av firkant-pulstog med 50\% duty-cycle og frekvens $f_\mathrm{ref} = 10 \mathrm{MHz}$.
    \item Måling av inngangssignal (referanse) ved kort kabel.
    \item Måling av utgangssignal etter transmisjonslinjen for ulike kabellengder.
    \item Kabel-lengder (1m, 10m, 50m). ????
    \item Gjentatte målinger for å vurdere konsistens og usikkerhet.
\end{itemize}



\subsection{Databehandling og estimering}
\paragraph{Frekvensrespons.}
Estimér $H_\text{målt}(\omega)$ fra tidsserier via FFT (vindustype, segmentlengde, overlap, Welch-averaging). De-embedd referanseledninger ved å dividere ut referansemåling.

\paragraph{Parameterestimering.}
Tilpass $R_0, L, C, \tan\delta$ ved å minimere
\[
\min_\theta \sum_{\omega\in\Omega} \left|\log H_\text{målt}(\omega) + \gamma(\theta;\omega)\,\ell\right|^2,
\]
og rapporter usikkerhet (bootstrap/repetisjoner + instrumentspesifikasjoner).

\subsection{Metrikker og akseptkriterier}
Vi sammenligner $|H(\omega)|$, fase/gruppeforsinkelse, risetid og oversving i tidsdomene. Akseptkriterier:
RMS-avvik $\leq X$ dB over $f_1 - f_2$, risetidsavvik $\leq Y$. Avvik diskuteres mot usikkerhetsbudsjettet.

\subsection{Reproduserbarhet}
Vi publiserer rådata, skript, faste parametre og seeds. Maskinvare/firmware og OS-versjoner oppgis.

\textcolor{red}{har kommentert ut de gamle forslagene, se om de er relevante for oss eller ikke \\ må isåfall integreres i teksten over på et vis}




\iffalse
\subsection{Numerisk simulering}
For å kunne presentere resultatene fra denne rapporten velger vi å simulere telegrafligningen numerisk. Dette gjøres ved å bruke en differensiallikning som beskriver spenningen og strømmen i en transmisjonslinje. Telegrafligningen kan uttrykkes som to koblede førsteordens differensiallikninger. \\ Koden for denne simuleringen lager et firkant-pulstog og deretter beregner Fourier-rekka numerisk. Hver harmoniske tilfelle av fourier-transformasjonen blir deretter sendt gjennom transmisjonslinjen ved å bruke overføringsfunksjonen. Signalet vi simulerer vil oppleve transmisjonstap med RLGC. Koden viser amplitudespekteret etter kabelen og tidsdomenesignalet etter ulike kabellengder, hhv. 10m, 50m og 100m. \\ koden er vedlagt i vedlegg \ref{appendix:code}. \textcolor{red}{SKAL VÆRE EN HENVENDELSE TIL VEDLEGG} \\ Koden er strukturert og fungerer følgende:



\subsection{Parameterstudie}

Dette avsnittet beskriver hvordan vi gjennomførte en parameterstudie for å undersøke effekten av de ulike parametrene i telegrafligningen på signaloverføring i en transmisjonslinje. Vi fokuserte på fire hovedparametere: motstand (R), induktans (L), konduktans (G), og kapasitans (C). Dette forsøket ble utført ved å variere hver parameter individuelt mens de andre ble holdt konstante, for å isolere deres spesifikke effekter på signaldemping og forvrengning. \\ Dette studiet ble utført ved hjelp av numeriske simuleringer i et verktøy som MATLAB eller Python. Vi modellerte en transmisjonslinje med en fast lengde og sendte et standardisert signal gjennom linjen, mens vi målte utgangssignalet for hver kombinasjon av parameterverdier. \\ \textcolor{red}{Vi startet med å definere en basislinje for hver parameter basert på typiske verdier for en standard Ethernet-kabel. Deretter varierte vi hver parameter over et realistisk spekter: DETTE I RØDT SKAL ENDRES TIL VÅRT FAKTISK FORSØK}

\subsection{Labaratoriestudie}

Labaratoriestudien ble utført ved å koble opp en funksjonsgenerator til et oscilloskop gjennom en kjent lengde av Ethernet-kabel. Vi brukte en Cat-5 kabel med spesifikasjoner som følger:
\begin{itemize}
    \item Motstand per lengdeenhet (R): 0.188 $\Omega/\mathrm{m}$
    \item Induktans per lengdeenhet (L): 0.25 $\mu \mathrm{H}/\mathrm{m}$
    \item Konduktans per lengdeenhet (G): 0.10 $\mathrm{mS}/\mathrm{m}$
    \item Kapasitans per lengdeenhet (C): 52 $\mathrm{pF}/\mathrm{m}$
\end{itemize}
Vi sendte et \textcolor{red}{type (kvadratisk, raskt?)} signal med en frekvens på \textcolor{red}{x Hz} fra en funksjonsgenerator gjennom en Ethernet-kabel og målte signalet ved mottakersiden med oscilloskopet. Vi registrerte både inngangs- og utgangssignalet for å analysere demping og forvrengning. Signalet ble sammenlignet med det av en direktekobling med koaksialkabel. Dette var for å ha en referanse for å vurdere effekten av transmisjonslinjen, og knytte den utledede teorien opp til praktiske observasjoner på labaratoriet. Resultatene fra målingene ble deretter sammenlignet med de numeriske simuleringene for å validere modellens nøyaktighet.
\fi