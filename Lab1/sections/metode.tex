\section{Metode}

\subsection{Mål og hypoteser.}
\textbf{Formål:} analysere hvordan en transmisjonslinje (koaksialkabel) påvirker et pulstog i tids- og frekvensdomenet, og validere en RLCG-modell basert på telegrafligningen, numerisk simulering og laboratoriemålinger. Den estimerte overføringsfunksjonen predikerer målt demping og fase innenfor forhåndsdefinerte akseptgrenser. Overføringsfunksjonen:

\[
H(\omega) = e^{-\gamma(\omega)\ell} = e^{-\sqrt{(R+j\omega L)(G+j\omega C)}\,\ell}
\]

\subsection{Oversikt}
Metoden vi bruker i denne rapporten består av to hoveddeler: numerisk simulering og laboratoriestudie. Den numeriske simuleringen innebærer å modellere transmisjonslinjen ved hjelp av telegrafligningen. Vi bruker Python for å implementere denne modellen og generere simuleringsdata. I laboratoriestudien gjennomfører vi praktiske målinger ved å sende et firkant-pulstog gjennom en koaksialkabel av kjent lengde og registrere inngangs- og utgangssignaler ved hjelp av et oscilloskop. Vi sammenligner deretter de målte dataene med de simulerte resultatene for å validere modellen.

\subsection{Numerisk simulering og modellering}
For å simulere telegrafligningen numerisk brukte vi Python med relevante biblioteker for numerisk beregning. Koden er vedlagt i vedlegg \ref{appendix:code}.
\paragraph{Modellantakelser.}
Vi modellerer kabelen som en lineær, tidsinvariant linje med fordelte parametre $R(f),L,C,G(f)$ per meter og bruker
\[
\gamma(\omega)=\sqrt{(R+j\omega L)(G+j\omega C)},\qquad
H(\omega)=e^{-\gamma(\omega)\,\ell}.
\]

\paragraph{Frekvensavhengigheter.}
Resistans skaleres med skinneffekt $R(f)$ (for $f>0$), og dielektriske tap modelleres ved
$G(f)$. Disse parameterne hentes/justeres fra datablad og brukes som startverdier i tilpasning.

\paragraph{Inngangssignal og representasjon.}
Vi modellerer pulstoget som en 50\% duty-cycle firkantpuls med periodetid $T=1/f_\mathrm{ref} = 100 \mathrm{ns}$ og grunnvinkelhastighet $\omega_0 = 2\pi / T$. Siden duty-cycle er 50\% består spekteret av kun oddetalls harmoniske. Tidsfunksjonen representeres med Fourier-rekken.
\begin{equation}
    V_{\mathrm{in}}(t) = \frac{5}{2} + \sum_{n=1,3,5,...}^{N_{\mathrm{harm}}}\frac{5}{jn\pi} e^{jn\omega_0 t}
\end{equation} 



\subsection{Laboratoriestudie}
\paragraph{Gjennomførelse}
Dette er vårt praktiske forsøk for å studere og analysere telegrafligningen i virkeligheten. Vi bruker en funksjonsgenerator til å generere et firkant-pulstog med 50\% duty-cycle og frekvens $f_\mathrm{ref} = 10 \mathrm{MHz}$, som sendes gjennom en koaksialkabel av kjente forskjellige lengder. På mottakersiden bruker vi et oscilloskop for å sammenlikne data av en kort, direktekoblet koaksialkabel (referanse) og gjennom en lengre koaksialkabel.

\paragraph{Hensikt.} Hensikten med laberatorieforsøket er å undersøke hvordan signalet endres når det overføres gjennom en transmisjonslinje over forskjellige lengder. Ved høyere frekvenser forventes amplitudetap, faseforskyvning og forvrengning av signalet, som vi ønsker å kvantifisere og analysere.

\paragraph{Utstyr og oppsett.}
Utstyret i forsøket er følgende:
\begin{itemize}
    \item \textbf{Funksjonsgenerator.} Genererer firkant-pulstog med justerbar frekvens og amplitude. Firkant-pulstoget ble konfigurert til periodetid $T = 2 \mu s$ (frekvens $f_\mathrm{ref} = 500 \mathrm{kHz}$) med 50\% duty-cycle og amplitude på $2.5V_{pp}$. Startfasen ble satt til $0\deg$ for å sikre at signalet starter ved null.
    \item \textbf{Oscilloskop.} FFT-funksjonalitet for frekvensanalyse.
    \item \textbf{Kabler.} Koaksialkabel med lengder på 5m og 30m.
\end{itemize}

\paragraph{Måleprosedyrer.}
Måleprosedyrene inkluderer kalibrering av funksjonsgenerator og oscilloskop, oppsett av firkant-pulstog, og måling av inngangs- og utgangssignaler for hver kabellengde. På oscilloskopet målte vi $V_{in}$ og $V_{out}$ for hver kabellengde, og registrerte dataene for videre analyse. Her benyttet vi oss også av FFT (Fast Fourier Transform) for å analysere frekvensspekteret til signalene. Vi gjennomfører gjentatte målinger for å vurdere konsistens og usikkerhet.

\subsubsection{FFT - Fast Fourier Transform}
For å analysere tidsseriene bruker vi Fast Fourier Transform (FFT) for å konvertere signalene fra tidsdomenet til frekvensdomenet. Dette gir oss muligheten til å studere frekvensresponsen til systemet mer detaljert. Oscilloskopet benytter derfor FFT internt for å vise frekvensspekteret av de målte signalene.
FFT er en effektiv algoritme for å beregne den diskrete Fourier-transformasjonen (DFT) og dens invers. Den reduserer beregningstiden betydelig sammenlignet med direkte beregning av DFT, spesielt for store datasett. FFT-algoritmen fungerer ved å dele opp en stor DFT i flere mindre DFT-er, som deretter kombineres for å gi det endelige resultatet. Dette gjør det mulig å analysere signaler raskt og effektivt, noe som er spesielt nyttig i sanntidsapplikasjoner som oscilloskopmålinger. \\ Vi benytter FFT fordi imotsetning til en mer vanlig algoritme som DFT, som har en beregningstid på $O(N^2)$, har FFT en beregningstid på $O(N \log N)$. Dette gjør FFT mye raskere og mer effektiv for store datasett, noe som er avgjørende for sanntidsanalyse av signaler i vårt eksperiment.