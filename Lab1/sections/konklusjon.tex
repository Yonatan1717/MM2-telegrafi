\section{Konklusjon}
Prosjektet har vist at telegrafligningen er en presis og anvendelig modell for å beskrive signalutbredning i transmisjonslinjer. Gjennom både numeriske simuleringer og laboratoriemålinger har vi sett på forskjellige faktorer som påvirker signalers form, amplitude og fase. Resultatene fra dette bekrefter at korte kabler bevarer signalformen godt, mens lengre kabler fører til tydelig demping av høyfrekvente signaler, og forvrengninger i tidsdomenet. Ved å se på både tids- og frekvensdomenet for simuleringene og laboratoriemålingene, har vi fått en bedre forståelse av hvordan signaler oppfører seg basert på telegrafligningen. Dette understøtter at telegrafligningen er et sentralt verktøy for å forstå hvordan elektriske signaler forplanter seg i transmisjonslinjer.
\\[1em]
Videre arbeid kan basere seg på å utforske mer komplekse transmisjonslinjer. Dette kan inkludere laboratorieanalyser av ethernetkabler, med effekten av tvinnede parledninger på signalet tatt i betraktning, samt ved å undersøke refleksjoner og tilpasning av impedanser. En problemstilling som kan være interessant å utforske videre er hvordan signaler i høyspentlinjer kan påvirkes av ytre faktorer, og hvordan dette kan føre til feil i strømnettet, ved å se på konsekvensene av de ytre faktorene med telegrafligningen.
\\[1em]
Avslutningsvis har prosjektet gitt en dypere forståelse av telegrafligningen og dens praktiske anvendelser i moderne kommunikasjonssystemer. Gjennom både teoretiske og praktiske tilnærminger har vi sett hvordan signaler oppfører seg i transmisjonslinjer, og hvordan ulike faktorer påvirker signalets kvalitet. Dette danner et solid grunnlag for videre studier innen signaloverføring og transmisjonslinjeteori.