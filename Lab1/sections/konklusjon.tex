\section{Konklusjon}
Telegrafligningen er en god modell for å beskrive signalutbredning i transmisjonslinjer. Vår analyse viser at de numeriske funnene basert på teorien, stemmer med funnene vi gjorde i laboratoriet. Dette indikerer at telegrafligningen gir en nøyaktig beskrivelse av signalutbredning i transmisjonslinjer.
\\[1em]
Videre arbeid kan basere seg på å utforske mer komplekse transmisjonslinjer. Dette kan inkludere labratorieanalyser av ethernetkabler, med effekten av tvinnede parledninger på signalet tatt i betraktning.
En problemstilling som kan være interessant å utforske videre er hvordan signaler i høyspentlinjer kan påvirkes av ytre faktorer, og hvordan dette kan føre til feil i strømnettet, ved å se på konsekvensene av de ytre faktorene med telegrafligningen.