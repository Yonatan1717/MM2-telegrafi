\section*{Sammendrag}
Denne rapporten analyserer signalforplantning i elektriske transmisjonslinjer med hovedvekt på tvinnet parkabel (Cat5e). Telegraflikningen utledes fra Kirchhoffs lover, og vi etablerer en frekvensavhengig RLCG-modell der \(R(f)\) beskriver skinneffekt og \(G(f)=\omega C\tan\delta\) dielektriske tap. Overføringsfunksjonen \(H(\omega)\), dempningskonstant \(\alpha(\omega)\), fasekonstant \(\beta(\omega)\) og gruppeforsinkelse \(\tau_g(\omega)=\mathrm{d}\beta/\mathrm{d}\omega\) brukes til å kvantifisere demping og dispersjon.\\[1em]
Vi gjennomfører numeriske simuleringer i Python for både tapsfri og realistisk Cat5e. En enkel laboratoriemåling på koaks (5 m og 30 m) brukes kun som \emph{prinsippbekreftelse} for å vise at høyfrekvente komponenter dempes sterkere og at pulser bredder seg og er ikke ment som verifikasjon av Cat5e-modellen.\\[1em]
Resultatene viser at korte lengder i hovedsak bevarer signalformen, mens økt lengde gir lavpass-karakteristikk, større \(\alpha(f)\) ved høye frekvenser og pulsbredding (økt stige-/falltid). Simuleringsfunnene er i kvalitativt samsvar med laboratorieobservasjonene og forklarer hvorfor standarder som Ethernet spesifiserer en øvre kabellengde.
