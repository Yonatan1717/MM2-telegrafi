\section*{Sammendrag}
<<<<<<< HEAD
Denne rapporten undersøker hvordan telegrafligningen kan brukes til å modellere signalforplantning i elektriske transmisjonslinjer. I denne rapporten ble det lagt spesiell vekt på forplantningen i koaksialkabler. Formålet har vært å forstå hvordan demping og dispersjon påvirker signalets form og kvalitet, spesielt når signalene må bevege seg over en lengre distanse. Vi har sammenlignet teoretiske modeller med praktiske målinger for å underbygge konklusjonen. 
\\
\\Arbeidet består av en teoretisk del hvor telegrafligningen utledes fra Kirchhoffs lover, og hvor begrepene demping, faseforskyvning og dispersjon analyseres ved hjelp av Fourier-metoder. Deretter ble en rekke numeriske modeller utviklet i Python for å simulere signalforløp. Dette ble vist gjennom modeller av transmisjonslinjer som spenner fra ideelle, tapsfrie tilfeller til realistiske modeller med både demping og dispersjon. Resultatene fra simuleringene ble sammenlignet med laboratoriemålinger av firkantpulser sendt gjennom koaksialkabler med lengdene 5m og 30m. Forvrengningene i disse kablene ble sammenliknet med en koaksialkabel på 1m.
\\
\\Resultatene viser at signalet i korte kabler i hovedsak bevarer formen sin, mens lengre kabler gir tydeligere demping av høyfrekvente komponenter og økt pulsbredding. Den teoretiske modellen basert på telegrafligningen beskriver disse effektene godt, og forklarer hvorfor praktiske standarder som Ethernet setter en øvre grense på kabellengde.
=======
Denne rapporten undersøker hvordan telegrafligningen kan brukes til å modellere signalforplantning i elektriske transmisjonslinjer. I rapporten ble det lagt spesiell vekt på forplantningen i koaksialkabler. Formålet har vært å forstå hvordan demping og dispersjon påvirker signalets form og kvalitet, spesielt når signalene må bevege seg over en lengre distanse. Vi har sammenlignet teoretiske modeller med praktiske målinger for å underbygge konklusjonen. 
Arbeidet består av en teoretisk del hvor telegrafligningen utledes fra Kirchhoffs lover, og hvor begrepene demping, faseforskyvning og dispersjon analyseres ved hjelp av Fourier-metoder. Deretter ble en rekke numeriske modeller utviklet i Python for å simulere signalforløp. Dette ble vist gjennom modeller av transmisjonslinjer som spenner fra ideelle, tapsfrie tilfeller til realistiske modeller med både demping og dispersjon. Resultatene fra simuleringene ble sammenlignet med laboratoriemålinger av firkantpulser sendt gjennom koaksialkabler med lengdene 5m og 30m. Forvrengningene i disse kablene ble sammenliknet med en koaksialkabel på 1m.
Resultatene viser at signalet i korte kabler i hovedsak bevarer formen sin, mens lengre kabler gir tydeligere demping av høyfrekvente komponenter og økt pulsbredding. Den teoretiske modellen basert på telegrafligningen beskriver disse effektene godt, og forklarer hvorfor praktiske standarder som Ethernet setter en øvre grense på kabellengde.
>>>>>>> 08baf8eebe9e2335bd2373c75e5256bee34ee985

